%this is the way to start the document.you can change the documents
%several settings by just changing the document class 
\documentclass{article}

%this is the way to include the packages
\usepackage{amsmath}
\usepackage{amssymb}
\usepackage{graphicx}

%a way to start the document, all the formatting of the document is
%performed before this step
\begin{document}

    %this is the way to create a title.this is the LaTeX's default tag
    %to set the title and its not displayed until you maketitle it
    \title{My tutorial on \LaTeX{} }
    \author{kaushal bondada}
    \maketitle

    \noindent \textbf{My first part on styles}

    
    %no indent to leave no space to start of the paragraph
    \noindent Hello world!!! \textbf{i am kaushal} \textit{i dont
    know what you are} \texttt{talking}

    \noindent \textbf{My second part on comments}


    %this is a comment. empty line is the way to create new paragraph
    \noindent this is not a comment

    \noindent \textbf{My third part on enumeration}


    %this is the way to create the lists and auto number all of
    %them
    %this command starts the list
    \begin{enumerate}
        %this is how you mark each item and explicitly number them
        %by placing in the square brackets
        \item[2.]  this is first
            %this is the way to create the sublists and similarly
            %you can explicitly specify
            \begin{enumerate}
                \item this is sublist one
                \item this is sublist two
            \end{enumerate}
        \item[3.] this is second
        \item[4.] this is third
    \end{enumerate}
    
    \noindent \textbf{My forth part on superscript and subscript}

    %this is a simple way to use "^" to super script and "_"
    %subscript the character after it.
    \noindent I am superscripting the folowing text $x^2$ and subscripting
    the text $x_y$ 

    %if you want to superscript or subscript a set of characters
    %after the command "^" or "_" then use the flower brackets

    \noindent now we shall work on slightly advanced part of super
    and sub scripting which is $x^{abd}$ and $x_{mno}$ and the most
    advanced $x^{abc}_{mno}$
    
    \noindent \textbf{My fifth part is on align}

    % align is used to align the equations based on "&" and prints
    % the numbers after each line break
    %align* does all same but do not print the equation numbers
    % & means "lineup here"`
    % \\ "line breaks"

    \noindent we would try to solve a equation $0=x^2+2x+1$
    \begin{align}
        0 &= x^2+2x+1 \\
          &= (x+1)(x+1) \\
          &= (x+1)^2    \\
          & \therefore  \\
        x &= 1 ,1      
    \end{align}
    \begin{align*}
        0 &= x^2+2x+1 \\
          &= (x+1)(x+1) \\
          &= (x+1)^2    \\
          & \therefore  \\
        x &= 1 ,1      
    \end{align*}

    \noindent \textbf{My sixth part is on pmatrix and bmatrix}

    % $ keeps every thing in line
    % pmatrix is for paranthesis matrix and bmatrix is for
    % bracketed matrix
    % & is for line up here. they need not follow a aligned as
    % mentioned in the example
    \noindent pmatrix
    $
    \begin{pmatrix}
        1 & 2 &3 \\
        4 &12344&6
    \end{pmatrix}
    $\\
    bmatrix
    $
    \begin{bmatrix}
        1 & 2 &3 \\
        4 &5&6
    \end{bmatrix}
    $

\end{document}

